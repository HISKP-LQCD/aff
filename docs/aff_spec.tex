\documentclass[10pt,letterpaper]{article}


\usepackage{color}
\definecolor{darkblue}{cmyk}{1,1,0,0.7}
\usepackage[dvipdfm,colorlinks=true,linkcolor=darkblue]{hyperref}
\usepackage{graphicx}
\usepackage{makeidx}
\renewcommand{\indexname}{\section*{INDEX}\addcontentsline{toc}{section}{\protect{\numberline{}INDEX}}}
\makeindex
%\usepackage{amsmath}


\newcommand{\bc}{\begin{center}}
\newcommand{\ec}{\end{center}}
\newcommand{\bi}{\begin{itemize}}
\newcommand{\ei}{\end{itemize}}             % terms and definitions
\newcommand{\term}[1]{\textit{#1}}          %
\newcommand{\ctext}[1]{\texttt{#1}}         % c mono text
\newcommand{\cvar}[1]{\ctext{#1}}           % c variables(in func. prototypes, etc.)
\newcommand{\libname}[1]{\ctext{#1}\index{\ctext{#1}}} % lib names, with indexing
\newcommand{\ctype}[1]{\ctext{#1}}          % c type names, no indexing
\newcommand{\libtype}[1]{\libname{#1}}      % lib type names, with indexing
\newcommand{\NULL}{\ctext{NULL}}


\newcommand{\exblock}[1]{\par\texttt{#1}\par}
\newcommand{\example}[1]{\texttt{#1}}

%\newcommand{\Description}{{\textbf Description}\\}
%\newcommand{\Synopsis}{{\textbf Synopsis}\\}
%\newcommand{\RetValue}{{\textbf Return value}\\}
\def\FuncHead#1{\subsubsection{\libname{#1()}}}
\def\Syn{\par\noindent{\bf Synopsis}\tt
  \everypar={\hangindent=64pt\hangafter=1
    \raggedright\hspace{16pt}}\par\noindent}
\def\Desc#1{\par\noindent{\bf Description}\par
  \hangindent=16pt\hangafter=0\noindent#1\par}
\def\RetVal#1{\par\noindent{\bf Return Value}\par
  \hangindent=16pt\hangafter=0\noindent#1\par}

\setcounter{tocdepth}{2}

\begin{document}
\centerline{\huge\bf AFF lattice data storage format}
\vspace{20pt}
\centerline{\it Andrew~V.~Pochinsky, Sergey~N.~Syritsyn}
\centerline{\it MIT CTP, Cambridge, MA}
\vspace{10pt}
\centerline{August 23, 2007}
\vspace{20pt}

\noindent This document describes the AFF data storage format. 
The AFF is hierarchical data format, efficient both in space and access time 
for storage of multiple small amounts of data.

\newpage
\tableofcontents
\newpage

\section{Purpose of AFF}
Lattice calculations produce a lot of data. 
This data usually consists of enormous number of small pieces, for example, correlator values for
each operator, each momentum and each in/out state.
Present approach is to store each piece\footnote{As it is done by ADAT stripping utilities}
as text in a separate file which has a fully descriptive name. 

Although this is both convenient for analysis and accessible by a text editor, text format leads
to a significant space overhead. 
In addition, some file systems, like PVFS, work badly with
many small files, taking too much time to open a file. 
Storing all data in an XML file leads to even greater space overhead; extraction of a single
data item requires parsing and validating the whole XML file.

AFF data storage format is aimed to replace this data storage scheme. 
AFF organization is aimed at optimization of data random read access.
We suggest to store all data related to a configuration or to an ensemble of configurations in
the same file.
To navigate an AFF data file, we introduce the system of hierarchical keys.
Data is stored in platform-independent, binary form.
To assure the data validity, both data and service information is checked against stored
MD5 checksum.



\section{Suggested AFF usage}

We suggest replacing the output of ADAT strippers with an AFF file format. 
Complicated file names will be replaced with hierarchical key names. 
The actual set of key names must be convenient for both interactive use and 
scripts for automatic data processing.

Scripts will access the data in an AFF file through command line utilities which will get the
data from a file and output the required data in appropriate (e.g., text) form. C/C++ analysis
codes will access the data through the AFF library, which will return data in the form
appropriate for a given platform.

Interactive browsing and modification of an AFF file will be done through command-line utilities. 
These utilities will allow searching and printing the keys, printing data, merging files,
insertion of data, and deleting entries. Probably, conversion from XML to ADAT will be possible,
but only for XML files with unique keys.

\section{Command line utility}
There is a command line utility, \libname{lhpc-aff} that allows one to manipulate AFF files from the shell, and has its own help.

\section{Platform-independent data}
Both data and service information is stored in platform-independent format. 
Information on bit size of numbers is written in the header of an AFF file.
All integer numbers are stored in big-endian form. 
Double precision numbers are stored in a portable binary format; the parameters of the floating
point representation are stored in the file providing enough information to restore \ctype{double}
numbers on a machine of any reasonable architecture. 
A complex number is stored as a sequence of two double precision numbers, first the real part,
and second the imaginary part.

\begin{table}[ht]
\bc
\caption{Numeric data types}\label{tab:types}
\begin{tabular}{|c|c|c|l|} \hline
Type & Size, bytes & Encoding & Comment \\ \hline
Void & 0 & \ctext{1}& Empty node \\
Char & 1 & \ctext{2}& String(array of chars) \\
Int   & 4 & \ctext{3}& 32-bit integer \\
Double & 8 & \ctext{4}& double precision real number \\
Complex & 16 & \ctext{5}& double precision complex number \\ \hline
\end{tabular}
\ec
\end{table}


\section{Data file organization}
An AFF file represents data organized as a tree structure. 
It starts at \term{root} key, which may have multiple subkeys. 
Each subkey of a given key must have a unique name. 

Each subkey may have data associated with it, which is an arbitrary length array of any
predefined elementary types.
Single number is represented as an array of length 1.
Possible data types are listed in table~\ref{tab:types}.

The data in AFF is named using hierarchical names, called \term{keys}. The namespace organization and semantics of the keys is very close to UNIX file names. A data key is a sequence of subkeys, which
we write here as a UNIX file name: \ctext{/key1/key2/.../keyN}. The top node in an AFF file is its
\term{root}, called \ctext{/}. Part of a key between consecutive slashes is called a \term{subkey}. To simplify transitions between AFF and XML, we restrict the character set allowed in subkeys is restricted to the following grammar (this is a subset of XML names):
\begin{eqnarray*}
\langle subkey \rangle & ::= & (\langle Letter \rangle | \verb|_| | \verb|:| ) | \langle nameChar \rangle^{*} \\
\langle nameChar \rangle & ::= & \langle Letter \rangle | \langle Digit \rangle | \verb|.| | \verb|-| | \verb|_| | \verb|:|
\end{eqnarray*}
The subkeys are case-sensitive as in XML.


\section{Data file layout}

An AFF file has:
\bi
\item a header, describing the numeric storage format, tables and data position and checksums; 
it is placed in the beginning of the file
\item a symbol table, storing all key names;
\item a tree table, storing all nodes of key tree
\item a data section.
\ei

Each section may be located anywhere in a file. 
Their positions are stored in a header. 
An AFF file starts with a header, then there is usually a data section, and symbol and tree tables 
are in the end of the file. It should be noted that this order of sections in the AFF file is not mandated, a file is arbitrary places section is valid (even if they overlap.)

\subsection{Header}

\begin{table}[h]
\bc
\caption{Header layout}\label{tab:header}
\begin{tabular}{|l|c|} \hline
 & Size, bytes \\ \hline\hline
Signature & 32 \\ \hline
Symbol table header & 32 \\ \hline
Tree header & 32 \\ \hline
Data header & 32 \\ \hline
Header MD5 sum & 16 \\ \hline\hline
Total & 144 \\ \hline
\end{tabular}
\ec
\end{table}

\begin{table}[h]
\bc
\caption{Signature layout}\label{tab:signature}
\begin{tabular}{|l|c|} \hline
 & Size, bytes \\ \hline\hline
File version string, null-terminated &  21\\ \hline
Bits in Char & 1 \\ \hline
Bits in Double & 1 \\ \hline
Bits in Double mantissa & 1 \\ \hline
Exponents in Double & 4 \\ \hline
Header size in bytes & 4 \\ \hline\hline
Total & 32 \\ \hline
\end{tabular}
\ec
\end{table}


\subsection{Symbol table, tree, and data headers}
All three section headers have the same format described in table~\ref{tab:shdr}. Section offset and size is stored in big-endian order regardless of the machine endianess.
\begin{table}[h]
\bc
\caption{Symbol table, tree, and data header layout}\label{tab:shdr}
\begin{tabular}{|l|c|} \hline
 & Size, bytes \\ \hline\hline
Offset  & 8 \\ \hline
Size in bytes & 8 \\ \hline
Section MD5 sum & 16 \\ \hline
Total & 32 \\ \hline
\end{tabular}
\ec
\end{table}

\subsection{Symbol table}
Symbol table is a list of strings separated by a null char. The string stored in the symbol table are implicitly numbered starting with zero. This ordering is used in the tree table below to refer to subkeys of the nodes..

\subsection{Tree table}
The AFF file tree is represented by a table of entries. Each entry describes one node in a tree. Nodes without data have type Void and are stored according to table~\ref{tab:void}. All other nodes are stored according to table~\ref{tab:tree_entry}. Types are encoded according to table~\ref{tab:types}. The root node is not stored in the tree table, as it always has itself as a parent and an empty name, and there is data stored in it. Other nodes are implicitly numbered starting with \ctext{1}. These numbers are used in the parent node fields to refer to node's parent. A proper tree table describes a tree, e.g.,~every node has a parent and there is no cycles.

\begin{table}[h]
\bc
\caption{Void tree entry layout}
\label{tab:void}
\begin{tabular}{|l|c|} \hline
 & Size, bytes \\ \hline\hline
Type & 1 \\ \hline
Parent node Id & 8 \\ \hline
Node name Id (ref. to symbol table) & 4 \\ \hline
Total & 13 \\ \hline
\end{tabular}
\ec
\end{table}

\begin{table}[h]
\bc
\caption{Non-vod tree entry layout}
\label{tab:tree_entry}
\begin{tabular}{|l|c|} \hline
 & Size, bytes \\ \hline\hline
Type & 1 \\ \hline
Parent node Id & 8 \\ \hline
Node name Id (ref. to symbol table) & 4 \\ \hline
Size of stored array & 4 \\ \hline
Offset of stored data & 8 \\ \hline
Total & 25 \\ \hline
\end{tabular}
\ec
\end{table}


\section{AFF library interface}
AFF library is written in C and can be used by including the library header file \verb|lhpc-aff.h|.
There is also lhpc-aff-config utility that allows one to obtain proper flags 
and libraries needed by AFF.
The library uses global names starting with aff in all case combinations. 
Not all such names may be described in the present specification.
It is illegal to rely on behavior of undescribed functions, data and types.

The data types used by the AFF library interface are listed in table~\ref{tab:opaque_types}. All structures are made opaque so that the interface serves as an abstraction barrier between the implementation and the application codes. The only exception is \ctype{struct AffMD5\_s}.

\begin{table}
\caption{AFF interface opaque types.}
\label{tab:opaque_types}
\begin{tabular}{|l|c|} \hline
\libtype{struct AffWriter\_s} & A handler of an AFF file opened for writing \\ \hline
\libtype{struct AffReader\_s} & A handler of an AFF file opened for reading \\ \hline
\libtype{struct AffTree\_s} & A handler of an AFF tree \\ \hline
\libtype{struct AffNode\_s} & A handler of an AFF tree node \\ \hline
\libtype{struct AffSTable\_s} & A handler of an AFF symbol table \\ \hline
\libtype{struct AffSymbol\_s} & A symbol created and stored by the symbol table \\ \hline
\libtype{struct AffMD5\_s} & MD5 sum state \\ \hline
\libtype{enum AffNodeType\_e} & Type of the data stored in a node \\ \hline
\end{tabular}
\end{table}

The interface consists of three parts.
\subsection{Library information}

\FuncHead{aff\_version}
{\Syn const char *aff\_version (void);\par}
\Desc{Returns a string identifying the library version.}
\RetVal{A non-\NULL\ string.}

\subsection{AFF Writer}
\FuncHead{aff\_writer}
{\Syn struct~AffWriter\_s~*aff\_writer (const~char~*fname);\par}
\Desc{Allocate a writer, and initialize it.
  Open a file for writing, initialize empty tables.
  If the file already exists, it is removed first.
  To query the status of \ctext{aff\_writer()} one calls
  \ctext{aff\_writer\_errstr()} on the result. 
  If \ctext{aff\_writer\_errstr()} returns \NULL, the object has been successfully
  created, otherwise \ctext{aff\_writer\_errstr()} returns a description of the error.
  Any pointer returned from \ctext{aff\_writer()} should be passed to 
  \ctext{aff\_writer\_close()} to free resources.
  }
\RetVal{Return a pointer to a \ctext{struct AffWriter\_s}. The status must be checked by calling
  \ctext{aff\_writer\_errstr()}.}


\FuncHead{aff\_writer\_close}
{\Syn const~char~*aff\_writer\_close (struct~AffWriter\_s~*aff);\par}
\Desc{Finalize writing, calculate MD5 sums, write all service tables and header, 
  and close the file.}
\RetVal{Return \NULL\ on success, and a pointer to an error string on failure.}

\FuncHead{aff\_writer\_errstr}
{\Syn const~char~*aff\_writer\_errstr (struct~AffWriter\_s~*aff);\par}
\Desc{Return a description of the error associated with the writer object.
  AFF implements latching errors: if an error occurs on a writer object, 
  this object will signal errors on all subsequent calls. 
  The first error message is stored in the object and is accessible 
  via \ctext{aff\_writer\_errstr()} call.}
\RetVal{Return the string describing the error recorded in the writer object, or \NULL\ if there were no errors.}

\FuncHead{aff\_writer\_stable}
{\Syn struct~AffSTable\_s~*aff\_writer\_stable (struct~AffWriter\_s~*aff);\par}
\Desc{Get the pointer the symbol table of the writer}
\RetVal{The pointer on success, or \NULL\ if the writer is not initialized.}

\FuncHead{aff\_writer\_tree}
{\Syn struct~AffTree\_s~*aff\_writer\_tree (struct~AffWriter\_s *aff);\par}
\Desc{Get the pointer to the tree table of the writer}
\RetVal{The pointer on success, or \NULL\ if the writer is not initialized.}
  
\FuncHead{aff\_writer\_root}
{\Syn struct~AffNode\_s~*aff\_writer\_root (struct~AffWriter\_s~*aff);\par}
\Desc{Get the handler to the root node. Any initialized writer always have a root node.}
\RetVal{The pointer on success, or \NULL\ if the writer is not initialized.}

\FuncHead{aff\_writer\_mkdir}
{\Syn struct~AffNode\_s~*aff\_writer\_mkdir (struct~AffWriter\_s~*aff, 
  struct~AffNode\_s~*dir, const~char~*name);\par}
\Desc{Create a new subkey \cvar{name} in the key node \cvar{dir} with type 
  \ctext{affNodeVoid} (no associated data type). The type may be changed later.}
\RetVal{Return the pointer to the new key node on success, and \NULL\ on failures,
  i.e. the writer is not initialized, the name already exists, or not 
  enough memory.}




%\FuncHead{aff\_node\_put\_char}
%\Syn{int aff\_node\_put\_char(struct AffWriter\_s *aff, struct AffNode\_s *n, const char *d,
%  uint32\_t s)}
%\Desc{Put a char array \cvar{d} of size \cvar{s} into AFF file \cvar{aff}, 
%  in the key node \cvar{n}.}
%\RetVal{Return zero on success, and non-zero on failure}
%
%\FuncHead{aff\_node\_put\_int}
%\Syn{int aff\_node\_put\_int(struct AffWriter\_s *aff, struct AffNode\_s *n, 
%  const uint32\_t *d, uint32\_t s)}
%\Desc{Put an int32 array \cvar{d} of size \cvar{s} into AFF file \cvar{aff}, 
%  in the key node \cvar{n}.}
%\RetVal{Return zero on success, and non-zero on failure}

\FuncHead{aff\_node\_put\_{\it type}}
{\Syn int~aff\_node\_put\_char (struct~AffWriter\_s~*aff,
  struct~AffNode\_s~*n, const~char~*d, uint32\_t~s);

  int~aff\_node\_put\_int (struct~AffWriter\_s~*aff,
  struct~AffNode\_s~*n, const~uint32\_t~*d, uint32\_t~s);
  
  int~aff\_node\_put\_double (struct~AffWriter\_s~*aff, 
  struct~AffNode\_s~*n, const~double~*d, uint32\_t~s);
  
  int~aff\_node\_put\_complex (struct~AffWriter\_s~*aff, 
  struct~AffNode\_s~*n, const~double~\_Complex~*d, uint32\_t~s); \par
}
\Desc{Put an array \cvar{d} of {\it type} of size \cvar{s} into AFF file \cvar{aff} 
  in the key node \cvar{n}. Type may be \ctype{char}, \ctype{int}(32 bits), 
  \ctype{double} or \ctype{complex}.}
\RetVal{Return zero on success, and non-zero on failure.}
%put a complex array ``d'' of size ``s'' into AFF file ``aff'' key node ``n'';

\subsection{AFF Reader}

\FuncHead{aff\_reader}
{\Syn struct~AffReader\_s~*aff\_reader (const~char~*file\_name);\par}
\Desc{Allocate a reader and initialize it. 
  Open a file for reading, read all tables. To check the status of \ctext{aff\_reader()}, 
  \ctext{aff\_reader\_errstr()} must be called. \ctext{aff\_reader\_errstr()} returns \NULL\ on success, 
  and the description of a problem otherwise. Any pointer returned by \ctext{aff\_reader()} 
  must be passed later to \ctext{aff\_reader\_close()} to free the resources.}
\RetVal{Return a pointer to \ctext{struct AffWriter\_s}. The status must be checked by calling
  \ctext{aff\_writer\_errstr()}.}

\FuncHead{aff\_reader\_close}
{\Syn void~aff\_reader\_close (struct~AffReader\_s~*aff);\par}
\Desc{Close a file, deallocate a reader and all its tables.}

\FuncHead{aff\_reader\_errstr}
{\Syn const~char~*aff\_reader\_errstr (struct~AffReader\_s~*aff);\par}
\Desc{Get an error string from the last failure.}
\RetVal{Return a pointer to a string, or \NULL\ if no errors have occurred.}

\FuncHead{aff\_reader\_stable}
{\Syn struct~AffSTable\_s~*aff\_reader\_stable (const~struct~AffReader\_s~*aff);\par}
\Desc{Get reader's symbol table.}
\RetVal{Return a pointer to the symbol table, or \NULL\ if \cvar{aff} 
  is not initialized.}
  
\FuncHead{aff\_reader\_tree}
{\Syn struct~AffTree\_s~*aff\_reader\_tree (struct~AffReader\_s~*aff);\par}
\Desc{Get the reader's tree table.}
\RetVal{Return a pointer to the symbol table, or \NULL\ if \cvar{aff} 
  is not initialized.}
  
\FuncHead{aff\_reader\_root}  
{\Syn struct~AffNode\_s~*aff\_reader\_root (struct~AffReader\_s~*aff);\par}
\Desc{Get the root node handler of the reader. Root node is always defined, even for empty tree.}
\RetVal{Return a pointer to the root node handler, or \NULL\ if \cvar{aff} 
  is not initialized.}
  
\FuncHead{aff\_reader\_chdir}
{\Syn struct~AffNode\_s~*aff\_reader\_chdir (struct~AffReader\_s~*aff, 
  struct~AffNode\_s~*dir, const~char~*name);\par}
\Desc{Get the handler to the subkey \cvar{name} in the key node \cvar{dir}.
  If the node does not exist, an error will be set in the reader object. Note that this function should not be
  used to probe for presence of a subkey because of failure it will render the reader unusable.}
\RetVal{Return a pointer to the handler or \NULL\ if it does not exist 
  or there is other failure.}

\FuncHead{aff\_node\_get\_{\it type}}
{\Syn int~aff\_node\_get\_char (const~struct~AffReader\_s~*aff, 
  const~struct~AffNode\_s~*n, char *d,~uint32\_t~s);\par
  int~aff\_node\_get\_int (const~struct~AffReader\_s~*aff, 
  const~struct~AffNode\_s~*n, int32\_t~*d, uint32\_t~s);\par
  int~aff\_node\_get\_double (const~struct~AffReader\_s~*aff, 
  const~struct~AffNode\_s~*n, double~*d, uint32\_t~s);\par
  int~aff\_node\_get\_complex (const~struct~AffReader\_s~*aff,
  const~struct~AffNode\_s~*n, double~\_Complex~*d, uint32\_t~s);\par}
\Desc{Get an array of {\it type} of size \cvar{s} from AFF file \cvar{aff} 
  in the key node \cvar{n} and store it to \cvar{d}. 
  Type may be \ctype{char}, \ctype{int}(32 bits),
  \ctype{double} or \ctype{complex}.
  If the data type does not match, an error will be set in the reader object. The size\cvar{s} may differ
  from the size of the node. If \cvar{s} is smaller than the node size, \cvar{d} will receive the initial portion of the node data. If \cvar{s} is larger than the node data, its initial portion will be filled with the node data. Values in the rest of the buffer are unspecified in this case.
  }
\RetVal{Return zero on success, and non-zero on failure. An failure causes an error to be stored in the reader object.}

\section{AFF low level interfaces}
The rest of AFF provides low level access to the library structures. Some of them are exported only because they are perceived to be generally useful, other are needed for non-trivial manipulation with
the AFF objects. Users are advised to treat the functions below with respect.

\subsection{AFF tree navigation}

\FuncHead{aff\_node\_foreach}
{\Syn void~aff\_node\_foreach (struct~AffNode\_s~*n,  
  void~(*proc)(struct~AffNode\_s~*child,  void~*arg),
  void~*arg);\par}
\Desc{Call function \cvar{proc} for each child of the node \cvar{n},
  and transfer \cvar{arg} as an argument. If \cvar{n} is \NULL\, nothing is done.}

\FuncHead{aff\_node\_id}
{\Syn uint64\_t~aff\_node\_id (const~struct~AffNode\_s~*tn);\par}
\Desc{Get 64-bit node ID}
\RetVal{Return the node ID. If \cvar{tn} is \NULL\, return special value with all bits set.}

\FuncHead{aff\_node\_name}
{\Syn const~struct~AffSymbol\_s~*aff\_node\_name (const~struct~AffNode\_s~*n);\par}
\Desc{Get the key name associated with the node}
\RetVal{Return pointer to a string containing key name. The string 
  is internal to the reader(writer) and must not be freed. If \cvar{n} is \NULL,
  return \NULL.}

\FuncHead{aff\_node\_parent}
{\Syn struct AffNode\_s~*aff\_node\_parent (const~struct~AffNode\_s~*n);\par}
\Desc{Get the handler of node's parent. The parent of the root node is the root itself.}
\RetVal{Return the pointer to the handler of parent node. If \cvar{n} is \NULL, return \NULL.}

\FuncHead{aff\_node\_type}
{\Syn enum~AffNodeType\_e~aff\_node\_type (const~struct~AffNode\_s~*n);\par}
\Desc{Determine the type of data stored in node \cvar{n}.}
\RetVal{Return type of data. If \cvar{n} is zero, return \cvar{affNodeInvalid}.}

\FuncHead{aff\_node\_size}
{\Syn uint32\_t~aff\_node\_size (const~struct~AffNode\_s~*n);\par}
\Desc{Get the size of the data array stored in the node \cvar{n}.}
\RetVal{Return size of data in data type units. Return zero if \cvar{n} is \NULL.}

\FuncHead{aff\_node\_offset}
{\Syn uint64\_t~aff\_node\_offset (const~struct~AffNode\_s~*tn);\par}
\Desc{Get the 64-bit file offset of the stored data of node \cvar{tn}. }
\RetVal{Return the byte offset of data. Return zero if \cvar{tn} is \NULL.}

\FuncHead{aff\_node\_assign}
{\Syn int~aff\_node\_assign (struct~AffNode\_s~*node,
  enum~AffNodeType\_e~type, uint32\_t~size, uint64\_t~offset);\par}
\Desc{Assign type to the node \cvar{node}. This function is internal to the library and should not
  be normally called by a user.}
\RetVal{Return zero on success, and non-zero on failure.}

\FuncHead{aff\_node\_{\it chdir}}
{\Syn struct~AffNode\_s~*aff\_node\_chdir (struct~AffTree\_s~*tree,
  struct~AffSTable\_s~*stable, struct~AffNode\_s~*n, int~create,
  const~char~*p);\par
  struct~AffNode\_s~*aff\_node\_cda (struct~AffTree\_s~*tree,
  struct~AffSTable\_s~*stable, struct~AffNode\_s~*n, int~create,
  const~char~*p[]);\par
  struct~AffNode\_s~*aff\_node\_cdv (struct~AffTree\_s~*tree,
  struct~AffSTable\_s~*stable, struct~AffNode\_s~*n, int~create,
  va\_list~va);\par
  struct~AffNode\_s~*aff\_node\_cd (struct~AffTree\_s~*tree,
  struct~AffSTable\_s~*stable, struct~AffNode\_s~*n, int~create, ...);\par
}
\Desc{aff\_node\_chdir returns the subkey of node \cvar{n} in the \cvar{tree} 
  with name \cvar{p}. aff\_node\_cda, aff\_node\_cdv, aff\_node\_cd descend the tree
  into subkeys with names transferred as \NULL-terminated array, va\_list 
  and \NULL-terminated argument list. If \cvar{create} is non-zero, all absent directories are
  created.}
\RetVal{Returns the handler of the target key on success. Returns \NULL\ if the target key is
  absent and \cvar{create} is zero, or attempt to create keys failed.}


\subsection{AFF tree data structure}

\FuncHead{aff\_tree\_init}
{\Syn struct~AffTree\_s~*aff\_tree\_init (void);\par}
\Desc{Allocate and initialize an AFF tree structure with only one node,
  which is root. The name of the root is an empty string ``''.}
\RetVal{Return a pointer to a new AFF tree, or \NULL\ if allocation failed.}

\FuncHead{aff\_tree\_fini}
{\Syn void~*aff\_tree\_fini (struct~AffTree\_s~*tree);\par}
\Desc{Free AFF data structure.}
\RetVal{Return \NULL. This helps with the following programming pattern:\\
  \ctext{tree\ =\ aff\_free\_fini(tree);}\\
  -- clean up the tree and guard stray accesses by setting it to \NULL.}

\FuncHead{aff\_tree\_foreach}
{\Syn void~aff\_tree\_foreach (const~struct~AffTree\_s~*tree,
  void~(*proc)(struct~AffNode\_s~*node, void~*arg), void~*arg);\par}
\Desc{Call function \cvar{proc} for each node of the tree in order of their ID numbers
  and pass \cvar{arg} as the argument. If \cvar{tree} is \NULL, nothing is done.}

\FuncHead{aff\_tree\_print}
{\Syn void~aff\_tree\_print (struct~AffTree\_s~*tree);\par}
\Desc{Print AFF tree for debug.}

\FuncHead{aff\_tree\_root}
{\Syn struct~AffNode\_s~*aff\_tree\_root (const~struct~AffTree\_s~*tree);\par}
\Desc{Get the root of the \cvar{tree}. A root is always present.}
\RetVal{Return a pointer to the root handler, or \NULL\ if \cvar{tree} is \NULL.}

\FuncHead{aff\_tree\_lookup}
{\Syn struct~AffNode\_s~*aff\_tree\_lookup (const~struct~AffTree\_s~*tree,
  const~struct~AffNode\_s~*parent, const~struct~AffSymbol\_s~*name);\par}
\Desc{Find the child of node \cvar{parent} with name \cvar{name}.}
\RetVal{Return a pointer to the child node handler, or \NULL\ if \cvar{tree} is \NULL
  or no such child is found.}

\FuncHead{aff\_tree\_index}
{\Syn struct~AffNode\_s~*aff\_tree\_index (const~struct~AffTree\_s~*tree,
  uint64\_t~index);\par}
\Desc{Get the node handler by its index. The index starts from zero, which is 
  reserved for the root node.}
\RetVal{Return a pointer to the node handler, or \NULL\ if \cvar{tree} is \NULL
  or no such node is found.}

\FuncHead{aff\_tree\_insert}
{\Syn struct~AffNode\_s~*aff\_tree\_insert (struct~AffTree\_s~*tree,
  struct~AffNode\_s~*parent, const~struct~AffSymbol\_s~*name);\par}
\Desc{Insert a child with name \cvar{name} to the node \cvar{parent}.}
\RetVal{Return a pointer to the new child node handler, or \NULL\ if such 
  node have already been present, \cvar{tree} is \NULL\ or the insertion failed.}

%\FuncHead{aff\_tree\_size}
%{\Syn uint64\_t~aff\_tree\_size (const~struct~AffTree\_s~*tn);\par}
%\Desc{Get the number of nodes, except the root.}
%\RetVal{Return the number of nodes except the root, or zero if \cvar{tn} is \NULL.}
%
%\FuncHead{aff\_tree\_file\_size}
%{\Syn uint64\_t~aff\_tree\_file\_size (const~struct~AffTree\_s~*tn);\par}
%\Desc{Get the size taken by the tree in an AFF file.}
%\RetVal{Return the size in bytes, or zero if \cvar{tn} is \NULL}
%

\subsection{AFF symbol table}

\FuncHead{aff\_stable\_init}
{\Syn struct~AffSTable\_s~*aff\_stable\_init (void);\par}
\Desc{Allocate and initialize an empty symbol table.}
\RetVal{Return a pointer to a new symbol table, or \NULL\ on failure.}

\FuncHead{aff\_stable\_fini}
{\Syn void~*aff\_stable\_fini (struct~AffSTable\_s~*st);\par}
\Desc{Free a symbol table.}

\FuncHead{aff\_stable\_print}
{\Syn void~aff\_stable\_print (const~struct~AffSTable\_s~*st);\par}
\Desc{Print symbol table for debug.}

\FuncHead{aff\_stable\_lookup}
{\Syn const~struct~AffSymbol\_s~*aff\_stable\_lookup (const~struct~AffSTable\_s~*st,
  const~char~*name);\par}
\Desc{Lookup a symbol in the table by its string name}
\RetVal{Return a pointer to symbol, or \NULL\ if there is no such symbol or
  \cvar{st} is zero.}

\FuncHead{aff\_stable\_index}
{\Syn const~struct~AffSymbol\_s~*aff\_stable\_index (const struct~AffSTable\_s~*st,
  uint32\_t~index);\par}
\Desc{Lookup a symbol in the table by its index. The index starts from zero.}
\RetVal{Return a pointer to the symbol, or \NULL\ if there is no such symbol or
  \cvar{st} is zero.}

\FuncHead{aff\_stable\_insert}
{\Syn const~struct~AffSymbol\_s~*aff\_stable\_insert (struct~AffSTable\_s~*st,
  const~char~*name);\par}
\Desc{Insert a new string into the symbol table. The string is duplicated insise.}
\RetVal{Return a pointer to the new symbol, or a pointer to the symbol with the same string
inserted before. Return \NULL\ if \cvar{st} is \NULL.}

%\FuncHead{aff\_stable\_size}
%{\Syn uint32\_t~aff\_stable\_size (const~struct~AffSTable\_s~*st);\par}
%\Desc{Get the number of symbols in the symbol table \cvar{st}.}
%\RetVal{Return the number of symbols in \cvar{st}, or zero if \cvar{st} is \NULL.}
%
%\FuncHead{aff\_stable\_file\_size}
%{\Syn uint64\_t~aff\_stable\_file\_size (const~struct~AffSTable\_s~*st);\par}
%\Desc{Get the size of symbol table if it is placed into AFF file.}
%\RetVal{Return the size in bytes, or zero if \cvar{st} is \NULL.}

\FuncHead{aff\_stable\_foreach}
{\Syn void~aff\_stable\_foreach (const~struct~AffSTable\_s~*st,
  void~(*proc)(const~struct~AffSymbol\_s~*sym, void~*arg), void~*arg);\par}
\Desc{Call the function \cvar{proc} for each symbol in the table in order of their 
  index passing \cvar{arg} as an argument. If \cvar{st} is zero, nothing is done.}

\subsection{AFF Symbols}

\FuncHead{aff\_symbol\_name}
{\Syn const~char~*aff\_symbol\_name (const~struct~AffSymbol\_s~*sym);\par}
\Desc{Get the name of the symbol. The string is stored internally in the symbol table
  and should not be freed or modified.}
\RetVal{Return a pointer to the null-terminated string, or \NULL\ if \cvar{sym} is \NULL.}

\FuncHead{aff\_symbol\_id}
{\Syn uint32\_t~aff\_symbol\_id (const~struct~AffSymbol\_s~*sym);\par}
\Desc{Get the index of a symbol.}
\RetVal{Return the index, or 0xffffffff if \cvar{sym} is zero.}

\subsection{Treap structure}

Manage treap data structure. \ctype{struct AffTreap\_s} is an opaque handler of a treap data structure.

\FuncHead{aff\_treap\_init}
{\Syn struct~AffTreap\_s~*aff\_treap\_init (void);\par}
\Desc{Allocate and initialize an empty treap.}
\RetVal{Return a pointer to a treap, or \NULL\ on failure.}

\FuncHead{aff\_treap\_fini}
{\Syn void~*aff\_treap\_fini (struct~AffTreap\_s~*h);\par}
\Desc{Free a treap.}

\FuncHead{aff\_treap\_cmp}
{\Syn int~aff\_treap\_cmp (const~void~*a\_ptr, unsigned~int~a\_size,
  const~void~*b\_ptr, unsigned~int~b\_size);\par}
\Desc{Compare key \cvar{a\_ptr} of length \cvar{a\_size} 
  with key \cvar{b\_ptr} of length \cvar{b\_size}. 
  This function defines the ordering used by
  the treap internaly. It is probably of little use to the user.}
\RetVal{Return \ctext{-1} if key \cvar{a\_ptr} is less than \cvar{b\_ptr}, 
  \ctext{+1} if key \cvar{a\_ptr} is greater than \cvar{b\_ptr},
  and zero if they are equal.}

\FuncHead{aff\_treap\_lookup}
{\Syn void~*aff\_treap\_lookup (const~struct~AffTreap\_s~*h,
  const~void~*key, int~ksize);\par}
\Desc{Lookup the the key \cvar{key} of length \cvar{ksize} in the treap \cvar{h}.}
\RetVal{Return the pointer to the data associated with the \cvar{key}, 
  or \NULL\ if there is no such key or \cvar{h} is \NULL.}

\FuncHead{aff\_treap\_insert}
{\Syn int~aff\_treap\_insert (struct~AffTreap\_s~*h,
  const~void~*key, int~ksize, void~*data);\par}
\Desc{Insert the pair \cvar{key} and \cvar{data} into the treap \cvar{h}. 
  Key must be unique.}
\RetVal{Return zero on successful insertion, or non-zero if the key is already present in the
  treap, insertion failed, or \cvar{h} is \NULL.}

\FuncHead{aff\_treap\_print}
{\Syn void~aff\_treap\_print (struct~AffTreap\_s~*h, 
  int~(*get\_vsize)(const~void~*));\par}
\Desc{Print the treap for debug.}


\section{MD5 sum functions}
This functions implement MD5 cryptographic checksum as described in RFC 1321. The implementation
is taken from the RFC, only the naming conventions were changed to confirm to the rest of the library.

\FuncHead{aff\_md5\_init}
{\Syn void~aff\_md5\_init (struct~AffMD5\_s~*);}
\Desc{Initialize MD5 sum state.}

\FuncHead{aff\_md5\_update}
{\Syn void~aff\_md5\_update (struct AffMD5\_s~*, const~uint8\_t~*, uint32\_t);}
\Desc{Update MD5 state when new data is added to a buffer.}

\FuncHead{aff\_md5\_final}
{\Syn void~aff\_md5\_final (uint8\_t~[16], struct~AffMD5\_s~*);}
\Desc{Produce the final value of MD5 sum.}

%features:
%  md5 checksum
%  portable, platform independent format
%  quick browsing, quick access to any stored data, without reading the whole file
%  operations: browsing, copy, merge, delete, formatted output
%  ?? how to change manually
%  ?? conversion from XML ONLY if all subkeys are unique
%  
%type table
%aff access interface
%  opaque types
%  open/close
%  put/get
%  insert node
%  ?? delete node
%suggested procedure:
%  strippers output an Aff file
%  analysis scripts access aff through command line
%  c/c++ code accesses through library
%motivation: slow access, enormous number of separate files by strippers
%
%
%
%FORMAT
%data (number) storage formats
%header formats
%
%header

%\section*{Index}

\printindex



\end{document}
